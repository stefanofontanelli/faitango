% ---------------------------------------------------------------------------- %
%  Description: Project Report of the FAI-TANGO Android App                    %
%                                                                              %
%  Author(s): chris                                                            %
%  Date: 31 March 2012 20.40                                                   %
% ---------------------------------------------------------------------------- %

\documentclass[10pt, twoside]{article}

% ----------------------------------------- %
% Including Packages                        %
% ----------------------------------------- %
\usepackage{fullpage} % To use 1 inch margins
%\usepackage{latex8}
%\usepackage{ieeetran}
\usepackage{color}
\usepackage{times}
\usepackage{url}
\usepackage{amsfonts}
\usepackage{amsmath}
\usepackage{graphicx}
%\usepackage{../../Styles/algorithm}
%\usepackage{../../Styles/algorithmic}
\usepackage{algorithm}
\usepackage{algorithmic}
\usepackage{theorem}
\usepackage{multirow}
\usepackage{epsf}
\usepackage{psfrag}
\usepackage{latexsym}
\usepackage{lscape}

% ----------------------------------------- %
% New commands and environments             %
% ----------------------------------------- %
\newcommand{\todo}[1]{{\color{red}{TODO: \textbf{\footnotesize{#1}}}}}

\newcommand{\app}{FAI-TANGO }

\renewcommand{\labelenumi}{\arabic{enumi}}
\renewcommand{\labelenumii}{\arabic{enumi}.\arabic{enumii}}
\renewcommand{\labelenumiii}{\arabic{enumi}.\arabic{enumii}.\arabic{enumiii}}

% ----------------------------------------- %
% Body (Title, author, abstract, sections)  %
% ----------------------------------------- %
%\graphicspath{{../figures/}}
%\DeclareGraphicsExtensions{.jpg}
%\DeclareGraphicsExtensions{.eps}
%\pagestyle{empty}
\title{\app Android App}
\author{Juri Lelli, Stefano Fontanelli, Christian Nastasi} 

\begin{document}
\maketitle
%\thispagestyle{empty}
% ----------------------------------------- %
% Abstract of the document                  %
% ----------------------------------------- %
%\begin{abstract} There is no abstract.  \end{abstract}

% ----------------------------------------- %
% Add document sections                     %
% ----------------------------------------- %
\small
\todo{Things to be made:
\begin{enumerate} 
\item Find a name for the App. I'm currently using \app
\end{enumerate} 
}
Topics:
\begin{itemize} 
    \item Introduction: about the app
    \item Requirements of the app
    \item General Architecture:
    \begin{itemize} 
        \item Local DB to store event information
        \item Module for information retrieval
        \item Module for information presentation
        \item Integration with Android: maps, calendar, facebook
    \end{itemize} 
    \item SW architecture: (basically the java packages) 
    \begin{itemize} 
        \item DB module
        \item GUI module
        \item Synch module 
        \item Preferences 
    \end{itemize} 
    \item DB module: DB helper, Content provider
    \item GUI module: main Activity, event detail activity
    \item Synch module: retrieval (EventFetcher and EventParser), ReaderService!
    \item Preferences
\end{itemize} 
\normalsize


\section{Application description}
\todo{Say something about: why we did this app?}

The \app application allows to retrieve and present information about
tango events registered at the \emph{FAItango} association 
(``Federazione delle Associazioni Italiane di Tango Argentino'').
Such information are available through the FAItango official web site,
but cannot be easily accessible on mobile handsets because of the lack of
a proper mobile version of the web site. 
Moreover, visualization through the web interface normally requires continuous
access to the internet.

The proposed application can be used to query the remote web site, store
events informations on the device and then display such informations in
a brief or detailed way. Additional features are event site visualization
through the Google Maps service, social sharing of an event details,
use of the Android integrated calendar to automatically add remainders.

\subsection*{Requirements}
\begin{enumerate}
    \item The tango event information is classified 
          in \emph{brief} and \emph{detailed}.
    \item Event whose starting date is previous than current date 
          (handset time) are considered as \emph{past}.
    \item The application shall retrieve the event information from the 
          FAItango database.
    \begin{enumerate}
        \item Brief and detailed information are accessible through 
              HTTP requests.
        \item Brief information are encoded in JSON standard.
        \item Detailed information are encoded in plain HTML.
        \item Brief and detailed information might be accessed and encoded using
              different mechanisms.
    \end{enumerate}
    \item The application shall present the event information on the handset.
    \begin{enumerate}
        \item The brief information shall be summarized in a sorted list.
        \begin{enumerate}
            \item The list is sorted by event date, closest to current date 
                  first
            \item Past events shall be excluded.
        \end{enumerate}
        \item The detailed information shall be presented upon selection of
              the event from the brief list.
    \end{enumerate}
    \item The application shall allow off-line visualization of the event 
          information.
    \begin{enumerate}
        \item The application shall store brief and detailed information 
              on the handset.
        \item Past events information should be removed.
    \end{enumerate}
    \item Information retrieval shall take place according to two possible 
          paradigms: \emph{synchronization} and \emph{search}.
    \item Synchronization
    \begin{enumerate}
        \item Synchronization shall accept search parameters to limit
              event information to be retrieved:
        \begin{enumerate}
            \item What info?...
        \end{enumerate}
        \item Synchronization shall be performed in one of the following mode:
              periodically, on-startup, manually.
        \begin{enumerate}
            \item Periodic synchronization shall occur according 
                  to predefined time interval (TBD: ....)
            \item Periodic synchronization shall occur even if the application
                  is not executing.
            \item On-startup synchronization shall occur when the application is
                  started.
            \item Manual synchronization shall occur on explicit request of the
                  user.
        \end{enumerate}
    \end{enumerate}
    \item Search
    \begin{enumerate}
        \item Describe...
    \end{enumerate}
    \item The application shall be integrated with: calendar, maps, social
          networks.
    \item The application shall provide off-line content to other possible
          application.
\end{enumerate}

\section{Architecture}
The architecture proposed to comply with the application requirements
is depicted in Figure~\ref{FIGURA}. 

TODO: mettere figura!


TODO: start describing...

\end{document}
